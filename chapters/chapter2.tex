\chapter{Heavily censored lifetime data}\label{chap:chapter2}

Computerised maintenance management systems (CMMS) such as SAP \citep{sap} are now embedded in companies maintenance procedures, meaning that these companies now posses large scale datasets of component installation and replacement times. A natural use of these personalised failure time data sets is for tailoring replacement strategies for the companies specific operating environments \citep[p. 13]{meeker2022}, rather than solely relying on the manufacturers recommendations. One problem however, is that these large observational datasets collected through CMMS are much messier that the experimental ones used by manufacturers in traditional reliability/warranty analysis. This messiness comes about because of reporting issues and the fact that most components are pre-emptively replaced before they fail because of the risk to production and employees safety. The result is that many of the valuable data sets stored in CMMS systems are heavily censored.

This heavy censoring results in biassed reliability estimates. Therefore, we propose a method of mitigating this bias so that we can still use the data to inform decisions.

In this chapter we\ldots

\section{Background}

Lifetime analysis, also called survival analysis, is the analysis of failure time data from a population of particular components/assets to derive the risk of failure of a component dependent on it's level of exposure (usually some form of time) and sometimes other covariates \citep{moore2016}. From here on we will use the general term unit/s to refer to individual/groups of the same asset or component. Lifetime analysis of a population of units typically takes place by first specifying a sampling distribution for the lifetimes by choosing some parametric lifetime distribution for the units and incorporating any observational characteristics of the data--for example censoring--then, estimating the parters of the distribution from failure time data using an appropriate inferential mechanism, and finally using the fitted model to derive useful reliability measures about the population which can be used to inform asset management plans. When done in a Bayesian context, the first step of this process also included specifying a prior distribution. From the resulting inference, we can devise optimal replacement strategies that minimise the risk of unplanned failures, and hence the risk of lost production.

\subsection{Lifetime distribution}

We model the lifetimes of the units as a random variable defined in terms of $t$, the exposure time, on $[0, \infty)$. $t$ is some continuous or discrete exposure time from a clearly defined origin, the installation of the component, to a well defined event, the failure of the component. 

These distributions are defined on a support. In reliability, the exposure is typically absolute time or the operating time of the unit. Say we choose a specific parametric lifetime distribution for the population of units, $p(t|\theta)$ expressed as the probability density function (CDF). Once we have estimated the parameters of the lifetime distribution we can draw useful interpretation of the

\begin{itemize}
    \item CDF (The probability that a unit will have failed by time $T$, i.e. $P(t <= T)$.)
    \item Survival function (The compliment of the CDF.)
    \item Hazard (The instantaneous failure rate at a given exposure.)
\end{itemize}

\subsection{The Weibull distribution}

Details of the Weibull.

\subsection{Censoring}

What is Censoring?

\subsection{An example from industry}

\section{Bias in heavily censored lifetime data}

Introduce the problem that, when data are heavily censored, the estimates of lifetime parameters become bias. We will show via simulated data so that the underlying truth is known.

\subsection{Simulation method}

How do we simulate censored lifetime data?

\subsection{Bias in results}

How does the estimated CDF differ from the truth?

\section{Informative Bayesian analysis}

How can informative priors help us in this case?

\paragraph*{Independent}

Construction of independent priors.

\paragraph*{Joint}

Construction of the joint prior.

\subsection{Effect of informative priors}

Compare the estimated CDF for the three different models with the truth.

\section{Discussion}

\ldots
%%
%% This is a file demonstrating the use of the abstract file in the Curtin thesis skeleton
%% file. And can be used as infrastructure to build your thesis.
%%

\chapter*{Abstract} \addcontentsline{toc}{chapter}{Abstract}

From pit to port, the consistent and efficient operation of iron ore machinery is essential for maximizing profits. To this end, reliability modelling is an invaluable tool for improving the design and execution of the maintenance strategies that ensure the reliable operation of mining machinery. There are well established reliability models in the literature, but there is a discrepancy between this literature and what is actually done by practitioners in the mining industry; a theory-practice gap. This gap exists because of the imperfect reality of collecting data in the field---data sets that are small, incomplete, noisy, or all three---and the lack of methods for expanding reliability modelling to account for these imperfections. My industry-linked PhD has aimed to reduce this gap by demonstrating how the Bayesian statistical modelling framework can address some of the common problems faced when fitting models to such reliability data in mining applications.

The first part of this works focuses on methods for analysing failure time data that is subject to right censoring and left truncation with unknown exposure history. I show how this incompleteness of lifetime data arises from the repeated replacements of a set of units. I propose a method for imputing partially observed left-truncated lifetimes through a Bayesian analysis. I also show how an informative joint prior for the two Weibull parameters can be carefully constructed to supplement the analysis in the case of censoring and truncation. I evaluate the methods using simulation and demonstrate on an industry dataset of idler-frame replacements. Using the output of the Bayesian analysis, I show how the failure times of idler-frames still in operation as well as the expected number of failures are easily obtained, and how inference about the parameters of the Weibull distribution can be used to inform the design of a fixed time replacement strategy.

In the second part of the work I focus on degradation modelling. Particularly, how the Bayesian hierarchical framework can extend the gamma stochastic degradation process to noisy observations and models for multiple units and then to the degradation of surfaces. In doing so, I simplify some of the literature on noisy gamma processes by demonstrating how separating the observation-degradation process into two separate conditional models removes the need for complicated inferential algorithms. Furthermore, I show the hierarchical models' implementations using flexible tools that are accessible to a wider reliability audience. I also show how reparametrisation can make the gamma process more interpretable and therefore simplify prior specification and clarify how the model can incorporate unit-to-unit variability (i.e. random effects). Taking this one step further, I combine the noisy gamma process with functional data analysis in order to model the degrading surface of conveyor belting.

Throughout the work I emphasise how complicated reliability processes found in practice can be broken down into manageable sub-models and how these models can be fit, evaluated, expanded, and compared using a Bayesian workflow considered to be good statistical practice. In doing so I hope to contribute at a larger level by providing an applied case study of the Bayesian workflow in a reliability setting that can be used as an exemplar by other applied reliability practitioners to develop solutions of their own for new problems.

\vspace*{\fill}